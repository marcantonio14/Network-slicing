\documentclass{article}
\usepackage[acronym,nomain]{glossaries}
\makeglossaries

\begin{document}

\tableofcontents
\newpage
\printglossaries
\newpage

%%%%%%%%%%%%%%%%%%%%%%%%%%%%%%%%%%%%%%%%%%%%%%%%%%%%%%%%%%%%%%%%%%%%%%%%%%%%%%%%%%%%%%%%%
% Acronym definitions
\newacronym{sdn}{SDN}{Software Defined Network}
\newacronym{mno}{MNO}{Mobile Network Operators}

\section{Introduction}
Mobile networks are a key element of today's society, enabling communication, access
and information sharing. Moreover, traffic forecasts predict that the
demand for capacity will grow exponentially over the next years, mainly due to
video services. However, as cellular networks move from being voice-centric to
data-centric, operators' revenues are not able to keep pace with the predicted
increase in traffic volume.Such pressure on operators' return on investment has
pushed research efforts toward designing for 5G novel mobile network solutions
able to open the door for new revenue sources. In this context, the network
slicing paradigm has emerged as a key SG disruptive technology addressing this
challenge.
Network slicing for 5G allows \gls{mno} to open
their physical network infrastructure platform to the concurrent deployment
of multiple logical self-contained networks, orchestrated in different ways according
to their specific service requirements; such network slices are then
(temporarily) owned by tenants. The availability of this vertical market multiplies
the monetization opportunities of the network infrastructure as (i) new
players may come into play (e.g., automotive industry, e-health) and (ii) a higher
infrastructure capacity utilization can be achieved by admitting network slice
requests and exploiting multiplexing gains.
With network slicing for 5G networks, different services (e.g., automotive,
mobile broadband, or haptic Internet) can be provided by different network slice
instances. Each of these instances consists of a set of virtual network functions
that run on the same infrastructure with a tailored orchestration. In this way,
very heterogeneous requirements can be provided on the same infrastructure, as
different network slice instances can be orchestrated and configured separately
according to their specific requirements. Additionally, this is performed in a
cost-efficient manner as the different network slice tenants share the same
physical infrastructure. \gls{sdn}
\newpage

\section{Fundamental parts}
bbbbb
\subsection{Modularization}
\subsection{SDN-VNF}
\subsection{Orchestration}

\newpage

\section{Network slicing}
\subsection{Services}
\subsection{Example}
\subsection{Actual realizations}
bbbbb


\newpage
\nocite*
\bibliographystyle{plain}
\bibliography{biblist}

\end{document}